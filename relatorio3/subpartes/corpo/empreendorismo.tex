
\section{Empreendedorismo}

O que é empreendedorismo?
\say{Empreendedorismo é a habilidade que uma pessoa tem para observar os problemas, identificar oportunidades, desenvolver soluções e investir em projetos com potencial de desenvolvimento.}

Não se nasce empreendedor, tornamos nos em empreendedores.

Esta é uma qualidade que se:
\begin{itemize}
      \item pode ser desenvolvida com formação prática.
      \item deriva da vontade
      \item é aplicada dentro e fora das organizações
      \item está cada vez mais presente fruto dos seus resultados positivos
\end{itemize}


\say{Empreendedorismo é algo que se pode nascer com e que se vai expandir com
estudo.}

\newpage
\subsection{Características de um empreendedor}


Este é sempre um inovador mas além disso:
\begin{itemize}
      \item Iniciativa e procura de oportunidades
      \item Persistência
      \item Correr risco calculados
      \item Exigência de qualidade e eficiência
      \item Comprometimento

      \say{É a responsablidade de assumir os louros do sucesso e as lágrimas do fracasso.}

      \item Procura de informações
      \item Estabelecimento de metas
      \item Planeamento e acompanhamento sistemáticos
      
      \say{Ouvir para intervir.}
      
      \item Persuasão e rede de contactos
      
      \say{Não ter medo, ter contactos, conhecer pessoas, ir a eventos.}

      \item Independência e autoconfiança
\end{itemize}



\subsection{Tipos de Empreendedores}



\begin{itemize}
      \item Inovadores
      
            São aqueles que apresentam ideias de novas em que são apaixonados Estes
            são os pecados pelos novos e os seus negócios

      \item Agressivo

            É alguém que trabalha e está disposto a sujar as mãos.

      \item Imitadores

            São aqueles que copiam uma ideia e tornam as melhores forma que ficou
            melhor no mercado

      \item Pesquisador

            Estes pesquisam bastante para tomar uma decisão, e não deixam espaço
            para erros

\end{itemize}
