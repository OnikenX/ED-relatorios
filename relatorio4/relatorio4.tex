\documentclass[a4paper,10pt]{report}
\usepackage[a4paper,width=150mm,top=25mm,bottom=25mm]{geometry}

\usepackage[utf8]{inputenc}
\usepackage{xspace}
\usepackage[portuguese]{babel}

% o comando say
\usepackage{dirtytalk}

%para que o primeiro paragrafo indente
\usepackage{indentfirst}

%Fontes especias de matemática
\usepackage{amsfonts}

%hide hilelinks esconde os links de terem bordas feias
\usepackage[hidelinks]{hyperref}
% pacote para os headers e footers fixes
\usepackage{fancyhdr}

%para poder ter os compactitem lists
\usepackage{paralist}

%ter acesso a imagens
\usepackage{graphicx}
\graphicspath{{images/}}

\usepackage{setspace}
\usepackage{afterpage}
% CORES

%tratar das cores
\usepackage{xcolor}
\usepackage{pagecolor,lipsum}% http://ctan.org/pkg/{pagecolor,lipsum}

% definição de uma cor
% \definecolor{name}{model}{color-spec}
\definecolor{aquamarine}{rgb}{0, 181, 190}




% meter o fundo em pretos e os caracters em branco
% \pagecolor{black}
% \color{white}
% \newcommand{\linkcolor}{aquamarine}
%link color for white mode
\newcommand{\linkcolor}{blue}




\usepackage{xcolor}
\usepackage{todonotes}
% \usepackage{biblatex}
% \addbibresource{referencias.bib}

%MACROS
\newcommand{\lk}[2]{\href{#1}{\textcolor{\linkcolor}{#2}}}
\newcommand{\chapterf}[1]{\chapter{#1}\thispagestyle{fancy}}


% Informações do autor
\newcommand{\myauthor}{João Pedro Neves Gonçalves}
\newcommand{\mynumerodealuno}{2018014306}

%valores da palestra
\newcommand{\palestratitulo} {O Colégio de Engenharia Informática:\\
                                oportunidades, domínios de intervenção e competências}
\newcommand{\palestraautor} {Isabel Pedrosa \& Jorge Sá Silva}
\newcommand{\palestradata} {7 de Abril de 2021}
\newcommand{\palestranumero} {4}


% set dos headers e footers
\pagestyle{fancy}
\fancyhf{}
\lhead{\mytitle{}}
\rhead {\thechapter}
\rfoot{\thepage}


%set do espaçamento entre linhas
\setstretch{1.25}
%set indentação
\setlength\parindent{0.75cm}

%set font family to arial
\usepackage{helvet}
\renewcommand{\familydefault}{\sfdefault}

\newcommand \mytitle {Relatório da Palestra nº \palestranumero}

\title{\mytitle}

\author{\myauthor}

\newcommand\myemptypage{
    \newpage
    \null
    \thispagestyle{empty}
    % \addtocounter{page}{-1}
    \newpage
    }


\begin{document}


\begin{titlepage}
    \begin{flushright}
        \textbf{Departamento de Engenharia Informática e de Sistemas
            Instituto Superior de Engenharia de Coimbra}

        Instituto Politécnico de Coimbra
    \end{flushright}
    \begin{center}
        \vspace{0.5cm}
        \begin{large}
            \newcommand{\spacetitle}{0.0cm}

            Curso Licenciatura Engenharia Informática - Diurno
            \vspace{\spacetitle}

            Ramo Desenvolvimento de Aplicações
            \vspace{\spacetitle}

            Unidade Curricular de Ética e Deontologia
            \vspace{\spacetitle}

            Ano Letivo de 2020/2021
            \vspace{\spacetitle}

            \vfill

            Palestra nº \palestranumero
            \vspace{\spacetitle}

            \palestratitulo
            \vspace{\spacetitle}

            Realizado por \palestraautor

            Realizado em \palestradata

            \vfill

            \includegraphics[width=0.4\textwidth]{ISEC}
            \vfill

            \myauthor

            Número de Aluno: \mynumerodealuno

            Coimbra, \today
            \vspace{\spacetitle}

            \myauthor

        \end{large}
    \end{center}
\end{titlepage}

\myemptypage{}

\tableofcontents{}
\thispagestyle{empty}

\newpage
\myemptypage{}
\newpage

\pagenumbering{arabic}
\section{Resumo}
Este trabalho é feito no âmbito de Ética e Deontologia, em que se faz um relatório sobre \palestratitulo que foi a \palestranumero ª palestra apresentada na disciplina.

\myemptypage{}

\chapterf{Introdução}

Por trás de cada empresa temos empreendedores e ideias de inovação para que estas empresas tenham exemplo, neste relatório é reportado o que foi apresentado na palestra sobres estes assuntos e também o exemplo de uma ideia, o Fablab que se mostrou um exemplo de sucesso. 




\myemptypage{}
\chapterf{Descrição da Palestra}

O tema envolveu se principalmente sobre Fab Lab, empreendedorismo, 
inovação e as ligações entre estes e a experiência dos palestrantes na area.

%analise

\section{Empreendedorismo}

O que é empreendedorismo?
\say{Empreendedorismo é a habilidade que uma pessoa tem para observar os problemas, identificar oportunidades, desenvolver soluções e investir em projetos com potencial de desenvolvimento.}

Não se nasce empreendedor, tornamos nos em empreendedores.

Esta é uma qualidade que se:
\begin{itemize}
      \item pode ser desenvolvida com formação prática.
      \item deriva da vontade
      \item é aplicada dentro e fora das organizações
      \item está cada vez mais presente fruto dos seus resultados positivos
\end{itemize}


\say{Empreendedorismo é algo que se pode nascer com e que se vai expandir com
estudo.}

\newpage
\subsection{Características de um empreendedor}


Este é sempre um inovador mas além disso:
\begin{itemize}
      \item Iniciativa e procura de oportunidades
      \item Persistência
      \item Correr risco calculados
      \item Exigência de qualidade e eficiência
      \item Comprometimento

      \say{É a responsablidade de assumir os louros do sucesso e as lágrimas do fracasso.}

      \item Procura de informações
      \item Estabelecimento de metas
      \item Planeamento e acompanhamento sistemáticos
      
      \say{Ouvir para intervir.}
      
      \item Persuasão e rede de contactos
      
      \say{Não ter medo, ter contactos, conhecer pessoas, ir a eventos.}

      \item Independência e autoconfiança
\end{itemize}



\subsection{Tipos de Empreendedores}



\begin{itemize}
      \item Inovadores
      
            São aqueles que apresentam ideias de novas em que são apaixonados Estes
            são os pecados pelos novos e os seus negócios

      \item Agressivo

            É alguém que trabalha e está disposto a sujar as mãos.

      \item Imitadores

            São aqueles que copiam uma ideia e tornam as melhores forma que ficou
            melhor no mercado

      \item Pesquisador

            Estes pesquisam bastante para tomar uma decisão, e não deixam espaço
            para erros

\end{itemize}

\newpage
\section{Inovação}

\say{Inovação a exploração com sucesso de novas ideias}

Inovações normalmente são ligadas à área tecnológica, ciência e marcado
Que significa inovar? 

Fazer coisas novas como renovar fazer arte
ciência, Inovação a trazer algo novo para o mercado, algo que se
disfarça os utilizadores e os consumidores, os empreendedores são
aqueles que conseguem efetuar com sucesso essas inovações.

Os vários tipos de Inovação são a inovação social, território e recursos,
estes três fazem parte da inovação de produtos que tem aplicações
comerciais e que tem um valor local, estes valores locais são ambientais
sociais e económicos.

Os grandes desafios da inovação, futuro é o bem-estar social, o amar o
território a paixão pelos filhos.

O virus obrigou é uma nova ordem social a atividade organizacional em
base de ações

\subsubsection{visa construir o quê?}

Clusters de territórios Redes Homens Compromissos Infraestruturas

\subsubsection{Making Health Safer}

Projeto que incentiva o empreendedorismo qualificado estimulando o
aparecimento de ideias inovadoras e a criação de novas empresas,
produtos e serviços com base na fabricação digital com potencial de
negócio pós-covid, promoção de saúde, novas plataformas de comércio e
serviços, turismo, entre outros setores económicos.

mobilizou os Fab Labs Em Portugal, faz parte de uma rede mundial de
Fab Labs

\subsubsection{Exemplo Prático (Projecto) Fablab}

Este é uma fabricação digital

O projeto Making Health Safer Visa incentivar o empreendedorismo
qualificado, estimulando o aparecimento de ideias inovadoras e a criação
de novas empresas, produtos e serviços com base na fabricação digital
com potencial de negócio pós-covid, soluções de prevenção.

A ideia veio do MIT E agora são quase 1750 laboratórios em todo o mundo.
existe uma grande distribuição os laboratórios em que se pode encontrar
tanto numa zona de grande densidade Urbana como numa floresta.

Tendo como objetivo partilhar informações tecnológicas abertamente.

Isto cria ritmos de Inovação nunca antes vistos.

\newpage
\subsubsection{reações imediatas}

Quantos exemplos que o Fab Lab fabricaram de modo extensivo: protetores
de ouvido, batentes e maçanetas de portas, caixas de incubação,
equipamento de proteção pessoal.

A nível global identificou-se o problema e reagiu se com a produção de:

\begin{compactitem}
      \item bolsa biológico
      \item máscara n-95
      \item coberturas
      \item batatas
      \item luvas
      \item camas de hospital
      \item respirador
      \item purificador
\end{compactitem}

O conceito de Fab Lab teve uma resposta bastante rápida em relação a
chegada do covid.

Atividades Making Health Safer.

Sensibilização.

Ignição Desafiar vários tipos de públicos que podem dar ideias de
negócio aceleração.



\subsection{fab lab reserva de ideias}

Todos tem o direito de declarar uma ideia como dele, mas tem de a
declarar dessa forma.


%critica
\chapterf{Análise Crítica}

\section{Apresentação}

Eu gostei muito da apresentação que o engenheiro Jorge fez, foi muito informativa e convivente para entrar na ordem, resumiu bem os assuntos e apresentou a um ótimo ritmo.



\myemptypage{}
\rhead{\thechapter}
\chapterf{Conclusão}


Foi uma ótima palestra sobre a ordem em que honestamente convenceu me a querer entrar nela pois para mim ter o apoio Jurídico e outros benefícios parecem ser bastante positivos pelos 10€ mensais.


\end{document}
