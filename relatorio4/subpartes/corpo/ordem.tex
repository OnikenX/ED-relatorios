
\section{Apresentação}
O engenheiro Jorge Sá Silva começa a palestra por apresentar varias questões que as pessoas fazem quando é para se inscrever na ordem dos engenheiros, estas são:

\begin{enumerate}
    \item Porquê inscrever-me na Ordem dos engenheiros?
    \item Porquê pagar os 10€ por mês?
    \item Porquê inscrever-me se a empresa que me vai contratar não exige isso?
    \item Acabo o curso e ainda tenho de fazer um estágio de vários meses sem ganhar dinheiro só para entrar para a OE?
    \item Devo inscrever-me na Ordem dos engenheiros Técnicos, sendo eu aluno de um Instituto Politécnico?
\end{enumerate}

E refere o que irá apresentar sobre a ordem:

\begin{enumerate}
    \item História
    \item Objetivos
    \item Organização e os seus membros
    \item O que fazem
    \item Benefícios
    \item Outros
\end{enumerate}

\newpage
\section{Historia}
\begin{itemize}
    \item Originou da Associação dos engenheiros Civis Portugueses
    \item Fundada em 1969
    \item O titulo de engenheiro foi definido pela primeira vez em 26 de Julho de 1926
    \item Foi formalizada em 1936
    \item A ordem é atualmente suportada por 12 colégios, em varias regiões do pais
\end{itemize}

\section{O que fazem}

O principal objetivo da ordem é contribuir com o progresso à Engenharia, tem a responsabilidade de atribuir o titulo oficial de engenheiro, é também está que sela os direitos e deveres dos seus membros.

Ela participa no ensino e desenvolvimento da engenheira como assistir no desenvolvimento da estrutura das cadeiras para os engenheiros, Desenvolvimento com outras instituições e apoiar projetos de éticos.

\section{A Organização}

A organização tem 9  órgãos nacionais e 5 regionais com vários conselhos e uma assembleia de representantes.

\section{Colégios}

\begin{itemize}
    \item Agronómica
    \item Ambiente
    \item Civil
    \item Eletrotécnica
    \item Florestal
    \item Geográfica
    \item Geológia e de Minas
    \item Informática
    \item Materiais
    \item Mecânica
    \item Naval
\end{itemize}

Estes podem ser vistos com mais detalhe 
\lk{https://www.ordemengenheiros.pt/pt/a-ordem/colegios-e-especialidades/}{neste site} 
\\
(https://www.ordemengenheiros.pt/pt/a-ordem/colegios-e-especialidades/).



\section{Membros}

Existem 2 tipos de membros:

\begin{itemize}
    \item Membros Estudantes
    \item Membros estagiários e efetivos
    \begin{itemize}
        \item Membros Efetivos
        \item Membros Seniores
        \item Membros Conselheiros
        \item Membros Honorários
    \end{itemize}
\end{itemize}

\section{Respondendo as questões iniciais}

Resumindo, a razão para entrar na ordem é porque esta instituição tem o cuidado de defender a profissão dos engenheiros, é uma ordem profissional/prestígio e é onde a sociedade obtém aconselhamento e assistência dentro da área. 

\section{Processo de admissão e qualificação}

A admissão é feita através da seguintes formas:

\begin{itemize}
    \item Se for do grau mestre pode fazer um estágio de 6 meses.
    \item Se for do grau licenciado pode fazer um estagio de 18 meses.
    \item Também pode fazer um curso de ética e deontologia profissional promovido pela Ordem dos engenheiros.
\end{itemize}

Notas:

\begin{itemize}
    \item Se já tiver experiência na área de 5 ou 6 anos poderá fazer dispensa do estagio.
    \item Normalmente os estágios são remunerados.
\end{itemize} 


\section{Níveis de qualificação}
\begin{itemize}
    \item Engenheira de nível 1 (licenciatura em engenharia)
    \item Engenheira de nível 2 (mestrado em engenharia)
\end{itemize}


\section{Prestigio/Benefícios}
\begin{itemize}
    \item Formação
    \item Seguro da responsabilidade Civil
    \item Informação Jurídica
    \item Revista da ordem dos engenheiros
    \item Bolsa de emprego
    \item Protocolos com diversas instituições e empresas que dão benefícios aos engenheiros
\end{itemize}

\section{Regalias para os membros}

\begin{itemize}
    \item Existem descontos em varias empresas.
    \item Eventos sociais
        \begin{itemize}
            \item A mulher na engenharia
            \begin{itemize}
                \item As Mulheres na Ciência
                \item Girls on IT
            \end{itemize}
            \item O melhor estágio
        \end{itemize}
\end{itemize}

\section{Colégio de Informática}
\begin{itemize}
    \item Um dos colégios
    \item 20.000 Engenheiros
\end{itemize}





\subsection{Atos de engenharia Informática}

Está dividida em 6 áreas:

\begin{itemize}
    \item Análise e validação de requisitos de soluções
    \item Programação de soluções
    \item Informáticas
    \item Integração de soluções
    \item Análise e validação de requisitos de infraestruturas de computação, comunicações e serviços
    \item Definição e modelação de arquiteturas de infraestruturas de computação, comunicações e serviços
    \item Definição e documentação de planos de gestão de projetos de sistemas de Informação
    \item Implementação de planos de gestão e auditoria de níveis de serviços, qualidade, risco e segurança em sistemas de informação
\end{itemize}

\section{Engenheiros pelo mundo}

Uma app implementada pelo colégio informático da zona centro, feita para descobrir os engenheiros espalhados pelo mundo.

